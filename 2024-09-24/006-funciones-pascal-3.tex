\section{Funciones en Pascal}

\begin{frame}
\frametitle{Funciones en Pascal}

\lstinputlisting[label=ejemplofuncion, caption=Ejemplo Funcion]{source_code/funcion.p}

\end{frame}
