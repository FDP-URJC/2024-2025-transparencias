\documentclass{beamer}
\usetheme{Darmstadt}
\usepackage{beamerthemesplit}
\usepackage{latexsym}
\usepackage{ae,aecompl}
\usepackage{graphicx}
\usepackage{amsfonts}
\usepackage{tikz}
\usepackage{graphics}
\usepackage{url}
\usepackage{listings}
\usepackage{courier}

\usepackage{caption}
\DeclareCaptionFont{white}{\color{white}}
\DeclareCaptionFormat{listing}{\colorbox[cmyk]{0.43, 0.35, 0.35,0.01}{\parbox{\textwidth}{\hspace{15pt}#1#2#3}}}
\captionsetup[lstlisting]{format=listing,labelfont=white,textfont=white, singlelinecheck=false, margin=0pt, font={bf,footnotesize}}

\graphicspath{{../images/}}

\lstset{
    basicstyle=\footnotesize\ttfamily, % Default font
    % numbers=left,              % Location of line numbers
    numberstyle=\tiny,          % Style of line numbers
    % stepnumber=2,              % Margin between line numbers
    numbersep=5pt,              % Margin between line numbers and text
    tabsize=2,                  % Size of tabs
    extendedchars=true,
    breaklines=true,            % Lines will be wrapped
    keywordstyle=\color{red},
    frame=b,
    % keywordstyle=[1]\textbf,
    % keywordstyle=[2]\textbf,
    % keywordstyle=[3]\textbf,
    % keywordstyle=[4]\textbf,   \sqrt{\sqrt{}}
    stringstyle=\color{white}\ttfamily, % Color of strings
    showspaces=false,
    showtabs=false,
    xleftmargin=17pt,
    framexleftmargin=17pt,
    framexrightmargin=5pt,
    framexbottommargin=4pt,
    % backgroundcolor=\color{lightgray},
    showstringspaces=false
}
\lstloadlanguages{ % Check documentation for further languages ...
     % [Visual]Basic,
     % C,
     % C++,
     % XML,
     % HTML,
     % Java,
     Pascal
}
\DeclareCaptionFont{white}{\color{white}}
\DeclareCaptionFormat{listing}{\colorbox[cmyk]{0.43, 0.35, 0.35,0.01}{\parbox{\textwidth}{\hspace{15pt}#1#2#3}}}
\captionsetup[lstlisting]{format=listing,labelfont=white,textfont=white, singlelinecheck=false, margin=0pt, font={bf,footnotesize}}

\newcommand*{\Scale}[2][4]{\scalebox{#1}{$#2$}}%

\hypersetup{
  pdftitle={Fundamentos de Programación (2024-09-24)},
  pdfauthor={Sergio Arroutbi Braojos},
  pdfcreator={},
  pdfproducer=PDFLaTeX,
  pdfsubject={nn},
}

\title{Fundamentos de Programación (2024-09-24)}
%\author{Sergio Arroutbi Braojos}

\begin{document}

\date{24 Septiembre, 2024}

\setbeamertemplate{headline}{}
\setbeamertemplate{frametitle}
{
  \nointerlineskip
  \begin{beamercolorbox}[sep=0.3cm,ht=1.8em,wd=\paperwidth]{frametitle}
  \vbox{}\vskip-2ex%
  \strut\insertframetitle\strut
  \vskip-0.8ex%
  \end{beamercolorbox}
}

% Background Image
\setbeamertemplate{background}{%
    \tikz\node[opacity=0.2] at (current page.center) {
    \vbox to \paperheight{\vfil\hbox to \paperwidth{\hfil\includegraphics[width=8cm]{urjc-logo-00.png}\hfil}\vfil}
 };

}

\frame{\titlepage
\begin{flushright}
{\tiny
(cc) 2024 Universidad Rey Juan Carlos.
    This work is under a license Creative Commons CC-BY 3.0.\\
To view a copy of full license, see \url{http://creativecommons.org/licenses/by/3.0/}
}
\begin{figure}[h]
    \begin{flushright}
        \includegraphics[width=0.6in]{by.png}
        \label{fig:by}
    \end{flushright}
\end{figure}
\end{flushright}
}

\section{Requisitos Iniciales}

\begin{frame}
\frametitle{Requisitos Iniciales}

\begin{itemize}
	\item Tener cuenta de Linux
	\item Tener acceso externo
	\item Conocer comandos básicos de Linux
	\begin{itemize}
		\item cat
		\item cd, mv, rm
		\item fpc 
		\item ls
		\item pwd
		\item gedit || code || vi || emacs ... (al menos un editor de texto disponible en Linux)
	\end{itemize}
	\item Saber editar y compilar
\end{itemize}

\end{frame}

\section{Objetivos}

\begin{frame}
\frametitle{Objetivos}

\begin{itemize}
	\item Introducción a las funciones como solución de problemas
	\item Terminar Ejercicio Práctico 1
\end{itemize}

\end{frame}

\section{Repaso}

\begin{frame}
\frametitle{Repaso}

\begin{itemize}
	\item Compilo con directiva de compilación predeterminada
	\item Los ficheros se nombran con extension (NEGRITA)".p"(NEGRITA)
	\item Tabulamos el código con tabuladores
	\item Utilizamos un layout común
\end{itemize}

\end{frame}

\section{Funciones}
\begin{frame}
\frametitle{Funciones}

\[\Scale[4]{f(x, y) = z}\]

\end{frame}

\section{Ejemplos de Funciones}

\begin{frame}
\frametitle{Ejemplos de Funciones}

\[\Scale[2]{suma(x, y) = x + y}\]
\[\Scale[2]{resta(x, y) = x - y}\]

\end{frame}

\section{Funciones en Pascal}

\begin{frame}
\frametitle{Funciones en Pascal}

\lstinputlisting[label=ejemplofuncion, caption=Ejemplo Funcion]{source_code/funcion.p}

\end{frame}

\section{Funciones}

\begin{frame}
\frametitle{Funciones}

\begin{itemize}
	\item Hacen lo que dicen
	\item Únicamente devuelven un valor
	\item Sirven para definir problemas / subproblemas
	\item Son sencillas. Si se complican, definimos otras funciones
\end{itemize}

\end{frame}

\section{Repaso}

\begin{frame}
\frametitle{Repaso}

\begin{itemize}
	\item Compilo con directiva de compilación predeterminada
	\item Los ficheros se nombran con extension \textbf{".p"}
	\item Identamos el código con tabuladores
	\item Utilizamos un layout común
	\item Funciones simples (poco código) que hacen lo que dicen
\end{itemize}

\end{frame}

\end{document}
