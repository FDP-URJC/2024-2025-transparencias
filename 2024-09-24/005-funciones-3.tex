\section{Funciones como subproblemas}

\begin{frame}
\frametitle{Funciones como subproblemas}

\begin{itemize}
\item \small{\textbf{<nombre del problema>(<datos necesarios>) <resultado>}}
\item La línea previa es la \textbf{cabecera de la función}. Debe ser obvio para cualquiera con el enunciado.
\item Una vez definido el problema:
    \begin{itemize}
    \item Si ya existe en Pascal: se usa la función de pascal, no se hace otra
    \item Si no existe en Pascal: se programa
    \end{itemize}
\end{itemize}

\end{frame}
